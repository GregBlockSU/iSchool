% Options for packages loaded elsewhere
\PassOptionsToPackage{unicode}{hyperref}
\PassOptionsToPackage{hyphens}{url}
%
\documentclass[
]{article}
\usepackage{amsmath,amssymb}
\usepackage{lmodern}
\usepackage{iftex}
\ifPDFTeX
  \usepackage[T1]{fontenc}
  \usepackage[utf8]{inputenc}
  \usepackage{textcomp} % provide euro and other symbols
\else % if luatex or xetex
  \usepackage{unicode-math}
  \defaultfontfeatures{Scale=MatchLowercase}
  \defaultfontfeatures[\rmfamily]{Ligatures=TeX,Scale=1}
\fi
% Use upquote if available, for straight quotes in verbatim environments
\IfFileExists{upquote.sty}{\usepackage{upquote}}{}
\IfFileExists{microtype.sty}{% use microtype if available
  \usepackage[]{microtype}
  \UseMicrotypeSet[protrusion]{basicmath} % disable protrusion for tt fonts
}{}
\makeatletter
\@ifundefined{KOMAClassName}{% if non-KOMA class
  \IfFileExists{parskip.sty}{%
    \usepackage{parskip}
  }{% else
    \setlength{\parindent}{0pt}
    \setlength{\parskip}{6pt plus 2pt minus 1pt}}
}{% if KOMA class
  \KOMAoptions{parskip=half}}
\makeatother
\usepackage{xcolor}
\usepackage[margin=1in]{geometry}
\usepackage{color}
\usepackage{fancyvrb}
\newcommand{\VerbBar}{|}
\newcommand{\VERB}{\Verb[commandchars=\\\{\}]}
\DefineVerbatimEnvironment{Highlighting}{Verbatim}{commandchars=\\\{\}}
% Add ',fontsize=\small' for more characters per line
\usepackage{framed}
\definecolor{shadecolor}{RGB}{248,248,248}
\newenvironment{Shaded}{\begin{snugshade}}{\end{snugshade}}
\newcommand{\AlertTok}[1]{\textcolor[rgb]{0.94,0.16,0.16}{#1}}
\newcommand{\AnnotationTok}[1]{\textcolor[rgb]{0.56,0.35,0.01}{\textbf{\textit{#1}}}}
\newcommand{\AttributeTok}[1]{\textcolor[rgb]{0.77,0.63,0.00}{#1}}
\newcommand{\BaseNTok}[1]{\textcolor[rgb]{0.00,0.00,0.81}{#1}}
\newcommand{\BuiltInTok}[1]{#1}
\newcommand{\CharTok}[1]{\textcolor[rgb]{0.31,0.60,0.02}{#1}}
\newcommand{\CommentTok}[1]{\textcolor[rgb]{0.56,0.35,0.01}{\textit{#1}}}
\newcommand{\CommentVarTok}[1]{\textcolor[rgb]{0.56,0.35,0.01}{\textbf{\textit{#1}}}}
\newcommand{\ConstantTok}[1]{\textcolor[rgb]{0.00,0.00,0.00}{#1}}
\newcommand{\ControlFlowTok}[1]{\textcolor[rgb]{0.13,0.29,0.53}{\textbf{#1}}}
\newcommand{\DataTypeTok}[1]{\textcolor[rgb]{0.13,0.29,0.53}{#1}}
\newcommand{\DecValTok}[1]{\textcolor[rgb]{0.00,0.00,0.81}{#1}}
\newcommand{\DocumentationTok}[1]{\textcolor[rgb]{0.56,0.35,0.01}{\textbf{\textit{#1}}}}
\newcommand{\ErrorTok}[1]{\textcolor[rgb]{0.64,0.00,0.00}{\textbf{#1}}}
\newcommand{\ExtensionTok}[1]{#1}
\newcommand{\FloatTok}[1]{\textcolor[rgb]{0.00,0.00,0.81}{#1}}
\newcommand{\FunctionTok}[1]{\textcolor[rgb]{0.00,0.00,0.00}{#1}}
\newcommand{\ImportTok}[1]{#1}
\newcommand{\InformationTok}[1]{\textcolor[rgb]{0.56,0.35,0.01}{\textbf{\textit{#1}}}}
\newcommand{\KeywordTok}[1]{\textcolor[rgb]{0.13,0.29,0.53}{\textbf{#1}}}
\newcommand{\NormalTok}[1]{#1}
\newcommand{\OperatorTok}[1]{\textcolor[rgb]{0.81,0.36,0.00}{\textbf{#1}}}
\newcommand{\OtherTok}[1]{\textcolor[rgb]{0.56,0.35,0.01}{#1}}
\newcommand{\PreprocessorTok}[1]{\textcolor[rgb]{0.56,0.35,0.01}{\textit{#1}}}
\newcommand{\RegionMarkerTok}[1]{#1}
\newcommand{\SpecialCharTok}[1]{\textcolor[rgb]{0.00,0.00,0.00}{#1}}
\newcommand{\SpecialStringTok}[1]{\textcolor[rgb]{0.31,0.60,0.02}{#1}}
\newcommand{\StringTok}[1]{\textcolor[rgb]{0.31,0.60,0.02}{#1}}
\newcommand{\VariableTok}[1]{\textcolor[rgb]{0.00,0.00,0.00}{#1}}
\newcommand{\VerbatimStringTok}[1]{\textcolor[rgb]{0.31,0.60,0.02}{#1}}
\newcommand{\WarningTok}[1]{\textcolor[rgb]{0.56,0.35,0.01}{\textbf{\textit{#1}}}}
\usepackage{graphicx}
\makeatletter
\def\maxwidth{\ifdim\Gin@nat@width>\linewidth\linewidth\else\Gin@nat@width\fi}
\def\maxheight{\ifdim\Gin@nat@height>\textheight\textheight\else\Gin@nat@height\fi}
\makeatother
% Scale images if necessary, so that they will not overflow the page
% margins by default, and it is still possible to overwrite the defaults
% using explicit options in \includegraphics[width, height, ...]{}
\setkeys{Gin}{width=\maxwidth,height=\maxheight,keepaspectratio}
% Set default figure placement to htbp
\makeatletter
\def\fps@figure{htbp}
\makeatother
\usepackage[normalem]{ulem}
\setlength{\emergencystretch}{3em} % prevent overfull lines
\providecommand{\tightlist}{%
  \setlength{\itemsep}{0pt}\setlength{\parskip}{0pt}}
\setcounter{secnumdepth}{-\maxdimen} % remove section numbering
\ifLuaTeX
  \usepackage{selnolig}  % disable illegal ligatures
\fi
\IfFileExists{bookmark.sty}{\usepackage{bookmark}}{\usepackage{hyperref}}
\IfFileExists{xurl.sty}{\usepackage{xurl}}{} % add URL line breaks if available
\urlstyle{same} % disable monospaced font for URLs
\hypersetup{
  pdftitle={R Notebook},
  hidelinks,
  pdfcreator={LaTeX via pandoc}}

\title{R Notebook}
\author{}
\date{\vspace{-2.5em}}

\begin{document}
\maketitle

\uline{\textbf{How to use an R Markdown Notebook}}

This is an \href{http://rmarkdown.rstudio.com}{R Markdown} Notebook.
When you execute code within the notebook, the results appear beneath
the code.

Try executing this chunk by clicking the \emph{Run} button within the
chunk or by placing your cursor inside it and pressing
\emph{Ctrl+Shift+Enter}.

To continue building our R notebook, we can add a new chunk by clicking
the \emph{Insert Chunk} button on the toolbar or by pressing
\emph{Ctrl+Alt+I}.

When you save the notebook, an HTML file containing the code and output
will be saved alongside it (click the \emph{Preview} button or press
\emph{Ctrl+Shift+K} to preview the HTML file).

The preview shows you a rendered HTML copy of the contents of the
editor. Consequently, unlike \emph{Knit}, \emph{Preview} does not run
any R code chunks. Instead, the output of the chunk when it was last run
in the editor is displayed.

\uline{\textbf{The Plot Surface}}

The R Markdown Notebook is an interactive surface that allows a viewer
to edit or execute calculations that can update the values displayed in
text, tables and plots. The surface is vertical, similar to a web page
that allows you to scroll vertically. This allows the visualization
designer to have a very large canvas in which to display plots, and also
imposes a linear order on the viewer. A drawback to this visual model is
that visualizations scroll off-screen when the viewer pages down. The
viewer cannot see a view of multiple visualizations at the same time.

\uline{\textbf{The examples}}

This first sample shows a single plot. Because there are no adjoining
plots, the plot appears on the left and right margins of the
notebook\ldots{} The plot is a scatterplot of the vehicle speed versus
its stopping distance.

\begin{Shaded}
\begin{Highlighting}[]
\FunctionTok{plot}\NormalTok{(cars)}
\end{Highlighting}
\end{Shaded}

\includegraphics{week-5-programming-lab-notebook_files/figure-latex/unnamed-chunk-1-1.pdf}

In this example, we again use mfrow to partition the plot surface into 2
rows and columns. Because the y axis is the same for all plots, we
suppress the label on the y axis and use a common title for all plots to
indicate that vehicle MPG is the common variable.

Note that the scale of the y axis is the same across all plots. Using
the same scale reduces the cognitive load on the audience, since they
can calculate the scale of the y axis once and carry that scale across
all the plots as they gaze through the panels.

\begin{Shaded}
\begin{Highlighting}[]
\FunctionTok{par}\NormalTok{(}\AttributeTok{mfrow=}\FunctionTok{c}\NormalTok{(}\DecValTok{2}\NormalTok{,}\DecValTok{2}\NormalTok{))}
\FunctionTok{boxplot}\NormalTok{(mpg }\SpecialCharTok{\textasciitilde{}}\NormalTok{ am, }\AttributeTok{data =}\NormalTok{ mtcars, }\AttributeTok{col =} \StringTok{"pink"}\NormalTok{, }\AttributeTok{main =} \StringTok{"Transmission"}\NormalTok{, }\AttributeTok{ylab=}\ConstantTok{NULL}\NormalTok{, }\AttributeTok{xlab=}\StringTok{"(Manual/Automatic)"}\NormalTok{)}
\FunctionTok{boxplot}\NormalTok{(mpg }\SpecialCharTok{\textasciitilde{}}\NormalTok{ cyl, }\AttributeTok{data =}\NormalTok{ mtcars, }\AttributeTok{col =} \StringTok{"yellow"}\NormalTok{, }\AttributeTok{main =} \StringTok{"Cylinders"}\NormalTok{, }\AttributeTok{ylab=}\ConstantTok{NULL}\NormalTok{, }\AttributeTok{xlab=}\StringTok{"(cylinder count)"}\NormalTok{)}
\FunctionTok{boxplot}\NormalTok{(mpg }\SpecialCharTok{\textasciitilde{}}\NormalTok{ gear, }\AttributeTok{data =}\NormalTok{ mtcars, }\AttributeTok{col =} \StringTok{"orange"}\NormalTok{, }\AttributeTok{main =} \StringTok{"Gears"}\NormalTok{, }\AttributeTok{ylab=}\ConstantTok{NULL}\NormalTok{, }\AttributeTok{xlab=}\StringTok{"(gear count)"}\NormalTok{)}
\FunctionTok{boxplot}\NormalTok{(mpg }\SpecialCharTok{\textasciitilde{}}\NormalTok{ hp, }\AttributeTok{data =}\NormalTok{ mtcars, }\AttributeTok{col =} \StringTok{"blue"}\NormalTok{, }\AttributeTok{main =} \StringTok{"Horsepower"}\NormalTok{, }\AttributeTok{ylab=}\ConstantTok{NULL}\NormalTok{, }\AttributeTok{xlab=}\StringTok{"(foot{-}pounds{-}per{-}second)"}\NormalTok{)}

\FunctionTok{mtext}\NormalTok{(}\SpecialCharTok{\textasciitilde{}} \FunctionTok{bold}\NormalTok{(}\StringTok{"Motor Trend Cars {-} MPG"}\NormalTok{), }\AttributeTok{side=}\DecValTok{3}\NormalTok{, }\AttributeTok{col=}\StringTok{"blue"}\NormalTok{, }\AttributeTok{line=}\SpecialCharTok{{-}}\FloatTok{1.5}\NormalTok{,}\AttributeTok{outer=}\ConstantTok{TRUE}\NormalTok{)}
\end{Highlighting}
\end{Shaded}

\includegraphics{week-5-programming-lab-notebook_files/figure-latex/unnamed-chunk-2-1.pdf}

\begin{Shaded}
\begin{Highlighting}[]
\FunctionTok{par}\NormalTok{(}\AttributeTok{mfrow=}\FunctionTok{c}\NormalTok{(}\DecValTok{1}\NormalTok{,}\DecValTok{1}\NormalTok{))}
\end{Highlighting}
\end{Shaded}

You can see in the plot surface below that the ggplot facet feature
works the same way it would in a conventional R document. The plot spans
the width of the page. In this plot we can see a faceted plot using the
iris species as the facet for 3 separate histograms.

\begin{Shaded}
\begin{Highlighting}[]
\FunctionTok{library}\NormalTok{(ggplot2)}
\FunctionTok{ggplot}\NormalTok{(iris, }\FunctionTok{aes}\NormalTok{(Sepal.Length, }\AttributeTok{fill =}\NormalTok{ Species)) }\SpecialCharTok{+}
  \FunctionTok{geom\_histogram}\NormalTok{(}\AttributeTok{bins =} \DecValTok{20}\NormalTok{) }\SpecialCharTok{+}
  \FunctionTok{scale\_fill\_viridis\_d}\NormalTok{() }\SpecialCharTok{+}
  \FunctionTok{facet\_wrap}\NormalTok{(}\SpecialCharTok{\textasciitilde{}}\NormalTok{ Species)}
\end{Highlighting}
\end{Shaded}

\includegraphics{week-5-programming-lab-notebook_files/figure-latex/unnamed-chunk-3-1.pdf}

Again, you can use the theme feature to paint the plot surface with
different fill colors. The plot below demonstrates the use of themes to
provide a rectangular background that spans the width of the page. As
always, it's best to be judicious with the use of theme colors, because
the user may print the plot and the page becomes saturated if the
background colors are overused. Also, it's better to remain
color-consistent with the foreground and background colors of all text
and plots in a notebook. Use themes sparingly, and for emphasis.

\begin{Shaded}
\begin{Highlighting}[]
\NormalTok{IrisPlot }\OtherTok{\textless{}{-}} \FunctionTok{ggplot}\NormalTok{(iris, }\FunctionTok{aes}\NormalTok{(Sepal.Length, Petal.Length, }\AttributeTok{colour=}\NormalTok{Species)) }\SpecialCharTok{+} 
  \FunctionTok{geom\_point}\NormalTok{() }\SpecialCharTok{+} 
  \FunctionTok{theme}\NormalTok{(}\AttributeTok{legend.title =} \FunctionTok{element\_text}\NormalTok{(}\AttributeTok{color =} \StringTok{"blue"}\NormalTok{, }\AttributeTok{size =} \DecValTok{10}\NormalTok{, }\AttributeTok{face =} \StringTok{"bold"}\NormalTok{)) }\SpecialCharTok{+}
  \FunctionTok{geom\_point}\NormalTok{(}\AttributeTok{color=}\StringTok{"firebrick"}\NormalTok{) }\SpecialCharTok{+}
  \FunctionTok{theme}\NormalTok{(}\AttributeTok{plot.background =} \FunctionTok{element\_rect}\NormalTok{(}\AttributeTok{fill =} \StringTok{\textquotesingle{}pink\textquotesingle{}}\NormalTok{))}
\FunctionTok{print}\NormalTok{(IrisPlot)}
\end{Highlighting}
\end{Shaded}

\includegraphics{week-5-programming-lab-notebook_files/figure-latex/unnamed-chunk-4-1.pdf}

This example generates two separate plots in the same code chunk when
viewed interactively in an R Markdown notebook. The notebook displays
the plots as thumbnails. You can click on either thumbnail, and the full
plot is displayed below. The two plots show iris species in a combined
density chart, as well as in facets for 3 separate density charts.

When exported to an HTML or PDF document, the two plots are stacked on
top of each other.

\begin{Shaded}
\begin{Highlighting}[]
\FunctionTok{ggplot}\NormalTok{(iris, }\FunctionTok{aes}\NormalTok{(}\AttributeTok{x =}\NormalTok{ Sepal.Length)) }\SpecialCharTok{+}
  \FunctionTok{geom\_density}\NormalTok{(}\FunctionTok{aes}\NormalTok{(}\AttributeTok{fill =}\NormalTok{ Species))}\SpecialCharTok{+}
  \FunctionTok{labs}\NormalTok{(}\AttributeTok{title=}\StringTok{"Combined"}\NormalTok{) }
\end{Highlighting}
\end{Shaded}

\includegraphics{week-5-programming-lab-notebook_files/figure-latex/unnamed-chunk-5-1.pdf}

\begin{Shaded}
\begin{Highlighting}[]
\FunctionTok{ggplot}\NormalTok{(iris, }\FunctionTok{aes}\NormalTok{(}\AttributeTok{x =}\NormalTok{ Sepal.Length)) }\SpecialCharTok{+}
  \FunctionTok{geom\_density}\NormalTok{(}\FunctionTok{aes}\NormalTok{(}\AttributeTok{color=}\NormalTok{Species)) }\SpecialCharTok{+}
  \FunctionTok{labs}\NormalTok{(}\AttributeTok{title=}\StringTok{"Faceted"}\NormalTok{) }\SpecialCharTok{+}
  \FunctionTok{facet\_wrap}\NormalTok{(}\SpecialCharTok{\textasciitilde{}}\NormalTok{Species)}
\end{Highlighting}
\end{Shaded}

\includegraphics{week-5-programming-lab-notebook_files/figure-latex/unnamed-chunk-5-2.pdf}

In this example, we use ggarange to split the plot surface into 2
columns. Rather than thumbnails, R notebook displays the plots
side-by-side. The two plots show iris species in a combined density
chart, as well as in facets for 3 separate density charts.

\begin{Shaded}
\begin{Highlighting}[]
\CommentTok{\#install.packages("ggpubr")}
\FunctionTok{library}\NormalTok{(ggpubr)}
\NormalTok{p1 }\OtherTok{\textless{}{-}} \FunctionTok{ggplot}\NormalTok{(iris, }\FunctionTok{aes}\NormalTok{(}\AttributeTok{x =}\NormalTok{ Sepal.Length)) }\SpecialCharTok{+}
  \FunctionTok{geom\_density}\NormalTok{(}\FunctionTok{aes}\NormalTok{(}\AttributeTok{fill =}\NormalTok{ Species))}\SpecialCharTok{+}
  \FunctionTok{labs}\NormalTok{(}\AttributeTok{title=}\StringTok{"Combined"}\NormalTok{) }

\NormalTok{p2 }\OtherTok{\textless{}{-}} \FunctionTok{ggplot}\NormalTok{(iris, }\FunctionTok{aes}\NormalTok{(}\AttributeTok{x =}\NormalTok{ Sepal.Length)) }\SpecialCharTok{+}
  \FunctionTok{labs}\NormalTok{(}\AttributeTok{title=}\StringTok{"Faceted"}\NormalTok{) }\SpecialCharTok{+}
  \FunctionTok{geom\_density}\NormalTok{(}\FunctionTok{aes}\NormalTok{(}\AttributeTok{color=}\NormalTok{Species)) }\SpecialCharTok{+}
  \FunctionTok{facet\_wrap}\NormalTok{(}\SpecialCharTok{\textasciitilde{}}\NormalTok{Species)}
\FunctionTok{ggarrange}\NormalTok{(p1, p2, }\AttributeTok{ncol =} \DecValTok{2}\NormalTok{, }\AttributeTok{nrow =} \DecValTok{1}\NormalTok{)}
\end{Highlighting}
\end{Shaded}

\includegraphics{week-5-programming-lab-notebook_files/figure-latex/unnamed-chunk-6-1.pdf}

\end{document}
